\documentclass[12pt,a4paper]{article}

\usepackage[a4paper,text={16.5cm,25.2cm},centering]{geometry}
\usepackage{lmodern}
\usepackage{amssymb,amsmath}
\usepackage{bm}
\usepackage{graphicx}
\usepackage{microtype}
\usepackage{hyperref}
\setlength{\parindent}{0pt}
\setlength{\parskip}{1.2ex}

\hypersetup
       {   pdfauthor = { Sheehan Olver },
           pdftitle={ foo },
           colorlinks=TRUE,
           linkcolor=black,
           citecolor=blue,
           urlcolor=blue
       }




\usepackage{upquote}
\usepackage{listings}
\usepackage{xcolor}
\lstset{
    basicstyle=\ttfamily\footnotesize,
    upquote=true,
    breaklines=true,
    breakindent=0pt,
    keepspaces=true,
    showspaces=false,
    columns=fullflexible,
    showtabs=false,
    showstringspaces=false,
    escapeinside={(*@}{@*)},
    extendedchars=true,
}
\newcommand{\HLJLt}[1]{#1}
\newcommand{\HLJLw}[1]{#1}
\newcommand{\HLJLe}[1]{#1}
\newcommand{\HLJLeB}[1]{#1}
\newcommand{\HLJLo}[1]{#1}
\newcommand{\HLJLk}[1]{\textcolor[RGB]{148,91,176}{\textbf{#1}}}
\newcommand{\HLJLkc}[1]{\textcolor[RGB]{59,151,46}{\textit{#1}}}
\newcommand{\HLJLkd}[1]{\textcolor[RGB]{214,102,97}{\textit{#1}}}
\newcommand{\HLJLkn}[1]{\textcolor[RGB]{148,91,176}{\textbf{#1}}}
\newcommand{\HLJLkp}[1]{\textcolor[RGB]{148,91,176}{\textbf{#1}}}
\newcommand{\HLJLkr}[1]{\textcolor[RGB]{148,91,176}{\textbf{#1}}}
\newcommand{\HLJLkt}[1]{\textcolor[RGB]{148,91,176}{\textbf{#1}}}
\newcommand{\HLJLn}[1]{#1}
\newcommand{\HLJLna}[1]{#1}
\newcommand{\HLJLnb}[1]{#1}
\newcommand{\HLJLnbp}[1]{#1}
\newcommand{\HLJLnc}[1]{#1}
\newcommand{\HLJLncB}[1]{#1}
\newcommand{\HLJLnd}[1]{\textcolor[RGB]{214,102,97}{#1}}
\newcommand{\HLJLne}[1]{#1}
\newcommand{\HLJLneB}[1]{#1}
\newcommand{\HLJLnf}[1]{\textcolor[RGB]{66,102,213}{#1}}
\newcommand{\HLJLnfm}[1]{\textcolor[RGB]{66,102,213}{#1}}
\newcommand{\HLJLnp}[1]{#1}
\newcommand{\HLJLnl}[1]{#1}
\newcommand{\HLJLnn}[1]{#1}
\newcommand{\HLJLno}[1]{#1}
\newcommand{\HLJLnt}[1]{#1}
\newcommand{\HLJLnv}[1]{#1}
\newcommand{\HLJLnvc}[1]{#1}
\newcommand{\HLJLnvg}[1]{#1}
\newcommand{\HLJLnvi}[1]{#1}
\newcommand{\HLJLnvm}[1]{#1}
\newcommand{\HLJLl}[1]{#1}
\newcommand{\HLJLld}[1]{\textcolor[RGB]{148,91,176}{\textit{#1}}}
\newcommand{\HLJLs}[1]{\textcolor[RGB]{201,61,57}{#1}}
\newcommand{\HLJLsa}[1]{\textcolor[RGB]{201,61,57}{#1}}
\newcommand{\HLJLsb}[1]{\textcolor[RGB]{201,61,57}{#1}}
\newcommand{\HLJLsc}[1]{\textcolor[RGB]{201,61,57}{#1}}
\newcommand{\HLJLsd}[1]{\textcolor[RGB]{201,61,57}{#1}}
\newcommand{\HLJLsdB}[1]{\textcolor[RGB]{201,61,57}{#1}}
\newcommand{\HLJLsdC}[1]{\textcolor[RGB]{201,61,57}{#1}}
\newcommand{\HLJLse}[1]{\textcolor[RGB]{59,151,46}{#1}}
\newcommand{\HLJLsh}[1]{\textcolor[RGB]{201,61,57}{#1}}
\newcommand{\HLJLsi}[1]{#1}
\newcommand{\HLJLso}[1]{\textcolor[RGB]{201,61,57}{#1}}
\newcommand{\HLJLsr}[1]{\textcolor[RGB]{201,61,57}{#1}}
\newcommand{\HLJLss}[1]{\textcolor[RGB]{201,61,57}{#1}}
\newcommand{\HLJLssB}[1]{\textcolor[RGB]{201,61,57}{#1}}
\newcommand{\HLJLnB}[1]{\textcolor[RGB]{59,151,46}{#1}}
\newcommand{\HLJLnbB}[1]{\textcolor[RGB]{59,151,46}{#1}}
\newcommand{\HLJLnfB}[1]{\textcolor[RGB]{59,151,46}{#1}}
\newcommand{\HLJLnh}[1]{\textcolor[RGB]{59,151,46}{#1}}
\newcommand{\HLJLni}[1]{\textcolor[RGB]{59,151,46}{#1}}
\newcommand{\HLJLnil}[1]{\textcolor[RGB]{59,151,46}{#1}}
\newcommand{\HLJLnoB}[1]{\textcolor[RGB]{59,151,46}{#1}}
\newcommand{\HLJLoB}[1]{\textcolor[RGB]{102,102,102}{\textbf{#1}}}
\newcommand{\HLJLow}[1]{\textcolor[RGB]{102,102,102}{\textbf{#1}}}
\newcommand{\HLJLp}[1]{#1}
\newcommand{\HLJLc}[1]{\textcolor[RGB]{153,153,119}{\textit{#1}}}
\newcommand{\HLJLch}[1]{\textcolor[RGB]{153,153,119}{\textit{#1}}}
\newcommand{\HLJLcm}[1]{\textcolor[RGB]{153,153,119}{\textit{#1}}}
\newcommand{\HLJLcp}[1]{\textcolor[RGB]{153,153,119}{\textit{#1}}}
\newcommand{\HLJLcpB}[1]{\textcolor[RGB]{153,153,119}{\textit{#1}}}
\newcommand{\HLJLcs}[1]{\textcolor[RGB]{153,153,119}{\textit{#1}}}
\newcommand{\HLJLcsB}[1]{\textcolor[RGB]{153,153,119}{\textit{#1}}}
\newcommand{\HLJLg}[1]{#1}
\newcommand{\HLJLgd}[1]{#1}
\newcommand{\HLJLge}[1]{#1}
\newcommand{\HLJLgeB}[1]{#1}
\newcommand{\HLJLgh}[1]{#1}
\newcommand{\HLJLgi}[1]{#1}
\newcommand{\HLJLgo}[1]{#1}
\newcommand{\HLJLgp}[1]{#1}
\newcommand{\HLJLgs}[1]{#1}
\newcommand{\HLJLgsB}[1]{#1}
\newcommand{\HLJLgt}[1]{#1}



\def\qqand{\qquad\hbox{and}\qquad}
\def\qqfor{\qquad\hbox{for}\qquad}
\def\qqas{\qquad\hbox{as}\qquad}
\def\half{ {1 \over 2} }
\def\D{ {\rm d} }
\def\I{ {\rm i} }
\def\E{ {\rm e} }
\def\C{ {\mathbb C} }
\def\R{ {\mathbb R} }
\def\H{ {\mathbb H} }
\def\Z{ {\mathbb Z} }
\def\CC{ {\cal C} }
\def\FF{ {\cal F} }
\def\HH{ {\cal H} }
\def\LL{ {\cal L} }
\def\vc#1{ {\mathbf #1} }
\def\bbC{ {\mathbb C} }



\def\fR{ f_{\rm R} }
\def\fL{ f_{\rm L} }

\def\qqqquad{\qquad\qquad}
\def\qqwhere{\qquad\hbox{where}\qquad}
\def\Res_#1{\underset{#1}{\rm Res}\,}
\def\sech{ {\rm sech}\, }
\def\acos{ {\rm acos}\, }
\def\asin{ {\rm asin}\, }
\def\atan{ {\rm atan}\, }
\def\Ei{ {\rm Ei}\, }
\def\upepsilon{\varepsilon}


\def\Xint#1{ \mathchoice
   {\XXint\displaystyle\textstyle{#1} }%
   {\XXint\textstyle\scriptstyle{#1} }%
   {\XXint\scriptstyle\scriptscriptstyle{#1} }%
   {\XXint\scriptscriptstyle\scriptscriptstyle{#1} }%
   \!\int}
\def\XXint#1#2#3{ {\setbox0=\hbox{$#1{#2#3}{\int}$}
     \vcenter{\hbox{$#2#3$}}\kern-.5\wd0} }
\def\ddashint{\Xint=}
\def\dashint{\Xint-}
% \def\dashint
\def\infdashint{\dashint_{-\infty}^\infty}




\def\addtab#1={#1\;&=}
\def\ccr{\\\addtab}
\def\ip<#1>{\left\langle{#1}\right\rangle}
\def\dx{\D x}
\def\dt{\D t}
\def\dz{\D z}
\def\ds{\D s}

\def\rR{ {\rm R} }
\def\rL{ {\rm L} }

\def\norm#1{\left\| #1 \right\|}

\def\pr(#1){\left({#1}\right)}
\def\br[#1]{\left[{#1}\right]}

\def\abs#1{\left|{#1}\right|}
\def\fpr(#1){\!\pr({#1})}

\def\sopmatrix#1{ \begin{pmatrix}#1\end{pmatrix} }

\def\endash{–}
\def\emdash{—}
\def\mdblksquare{\blacksquare}
\def\lgblksquare{\blacksquare}
\def\scre{\E}
\def\mapengine#1,#2.{\mapfunction{#1}\ifx\void#2\else\mapengine #2.\fi }

\def\map[#1]{\mapengine #1,\void.}

\def\mapenginesep_#1#2,#3.{\mapfunction{#2}\ifx\void#3\else#1\mapengine #3.\fi }

\def\mapsep_#1[#2]{\mapenginesep_{#1}#2,\void.}


\def\vcbr[#1]{\pr(#1)}


\def\bvect[#1,#2]{
{
\def\dots{\cdots}
\def\mapfunction##1{\ | \  ##1}
	\sopmatrix{
		 \,#1\map[#2]\,
	}
}
}



\def\vect[#1]{
{\def\dots{\ldots}
	\vcbr[{#1}]
} }

\def\vectt[#1]{
{\def\dots{\ldots}
	\vect[{#1}]^{\top}
} }

\def\Vectt[#1]{
{
\def\mapfunction##1{##1 \cr} 
\def\dots{\vdots}
	\begin{pmatrix}
		\map[#1]
	\end{pmatrix}
} }

\def\addtab#1={#1\;&=}
\def\ccr{\\\addtab}

\def\questionequals{= \!\!\!\!\!\!{\scriptstyle ? \atop }\,\,\,}

\begin{document}

\section{A Asymptotics and Computational Cost}
We introduce Big-O, little-o and asymptotic notation and see how they can be used to describe computational cost.

\begin{itemize}
\item[1. ] Asymptotics as $n \ensuremath{\rightarrow} \ensuremath{\infty}$


\item[2. ] Asymptotics as $x \ensuremath{\rightarrow} x_0$


\item[3. ] Computational cost

\end{itemize}
\subsection{1. Asymptotics as $n \ensuremath{\rightarrow} \ensuremath{\infty}$}
Big-O, little-o, and "asymptotic to" are used to describe behaviour of functions at infinity. 

\textbf{Definition (Big-O)} 

\[
f(n) = O(\ensuremath{\phi}(n)) \qquad \hbox{(as $n \ensuremath{\rightarrow} \ensuremath{\infty}$)}
\]
means

\[
\left|{f(n) \over \ensuremath{\phi}(n)}\right|
\]
is bounded for sufficiently large $n$. That is, there exist constants $C$ and $N_0$ such  that, for all $n \geq N_0$, $|{f(n) \over \ensuremath{\phi}(n)}| \leq C$.

\textbf{Definition (little-O)} 

\[
f(n) = o(\ensuremath{\phi}(n)) \qquad \hbox{(as $n \ensuremath{\rightarrow} \ensuremath{\infty}$)}
\]
means

\[
\lim_{n \ensuremath{\rightarrow} \ensuremath{\infty}} {f(n) \over \ensuremath{\phi}(n)} = 0.
\]
\textbf{Definition (asymptotic to)} 

\[
f(n) \ensuremath{\sim} \ensuremath{\phi}(n) \qquad \hbox{(as $n \ensuremath{\rightarrow} \ensuremath{\infty}$)}
\]
means

\[
\lim_{n \ensuremath{\rightarrow} \ensuremath{\infty}} {f(n) \over \ensuremath{\phi}(n)} = 1.
\]
\textbf{Examples}

\[
{\cos n \over n^2 -1} = O(n^{-2})
\]
as

\[
\left|{{\cos n \over n^2 -1} \over n^{-2}} \right| \leq \left| n^2 \over n^2 -1 \right|  \leq 2
\]
for $n \geq N_0 = 2$.

\[
\log n = o(n)
\]
as

\[
lim_{n \ensuremath{\rightarrow} \ensuremath{\infty}} {\log n \over n} = 0.
\]
\[
n^2 + 1 \ensuremath{\sim} n^2
\]
as

\[
{n^2 +1 \over n^2} \ensuremath{\rightarrow} 1.
\]
Note we sometimes write $f(O(\ensuremath{\phi}(n)))$ for a function of the form $f(g(n))$ such that $g(n) = O(\ensuremath{\phi}(n))$.

\subsubsection{Rules}
We have some simple algebraic rules:

\textbf{Proposition (Big-O rules)}


\begin{align*}
O(\ensuremath{\phi}(n))O(\ensuremath{\psi}(n)) = O(\ensuremath{\phi}(n)\ensuremath{\psi}(n))  \qquad \hbox{(as $n \ensuremath{\rightarrow} \ensuremath{\infty}$)} \\
O(\ensuremath{\phi}(n)) + O(\ensuremath{\psi}(n)) = O(|\ensuremath{\phi}(n)| + |\ensuremath{\psi}(n)|)  \qquad \hbox{(as $n \ensuremath{\rightarrow} \ensuremath{\infty}$)}.
\end{align*}
\subsection{2. Asymptotics as $x \ensuremath{\rightarrow} x_0$}
We also have Big-O, little-o and "asymptotic to" at a point:

\textbf{Definition (Big-O)} 

\[
f(x) = O(\ensuremath{\phi}(x)) \qquad \hbox{(as $x \ensuremath{\rightarrow} x_0$)}
\]
means

\[
|f(x) \over \ensuremath{\phi}(x)|
\]
is bounded in a neighbourhood of $x_0$. That is, there exist constants $C$ and $r$ such  that, for all $0 \leq |x - x_0| \leq r$, $|{f(x) \over \ensuremath{\phi}(x)}| \leq C$.

\textbf{Definition (little-O)} 

\[
f(x) = o(\ensuremath{\phi}(x)) \qquad \hbox{(as $x \ensuremath{\rightarrow} x_0$)}
\]
means

\[
\lim_{x \ensuremath{\rightarrow} x_0} {f(x) \over \ensuremath{\phi}(x)} = 0.
\]
\textbf{Definition (asymptotic to)} 

\[
f(x) \ensuremath{\sim} \ensuremath{\phi}(x) \qquad \hbox{(as $x \ensuremath{\rightarrow} x_0$)}
\]
means

\[
\lim_{x \ensuremath{\rightarrow} x_0} {f(x) \over \ensuremath{\phi}(x)} = 1.
\]
\textbf{Example}

\[
\exp x = 1 + x + O(x^2) \qquad \hbox{as $x \ensuremath{\rightarrow} 0$}
\]
Since

\[
\exp x = 1 + x + {\exp t \over 2} x^2
\]
for some $t \in [0,x]$ and

\[
\left|{{\exp t \over 2} x^2 \over x^2}\right| \leq {3 \over 2}
\]
provided $x \leq 1$.

\subsection{3. Computational cost}
We will use Big-O notation to describe the computational cost of algorithms. Consider the following simple sum

\[
\sum_{k=1}^n x_k^2
\]
which we might implement as:


\begin{lstlisting}
(*@\HLJLk{function}@*) (*@\HLJLnf{sumsq}@*)(*@\HLJLp{(}@*)(*@\HLJLn{x}@*)(*@\HLJLp{)}@*)
    (*@\HLJLn{n}@*) (*@\HLJLoB{=}@*) (*@\HLJLnf{length}@*)(*@\HLJLp{(}@*)(*@\HLJLn{x}@*)(*@\HLJLp{)}@*)
    (*@\HLJLn{ret}@*) (*@\HLJLoB{=}@*) (*@\HLJLnfB{0.0}@*)
    (*@\HLJLk{for}@*) (*@\HLJLn{k}@*) (*@\HLJLoB{=}@*) (*@\HLJLni{1}@*)(*@\HLJLoB{:}@*)(*@\HLJLn{n}@*)
        (*@\HLJLn{ret}@*) (*@\HLJLoB{=}@*) (*@\HLJLn{ret}@*) (*@\HLJLoB{+}@*) (*@\HLJLn{x}@*)(*@\HLJLp{[}@*)(*@\HLJLn{k}@*)(*@\HLJLp{]}@*)(*@\HLJLoB{{\textasciicircum}}@*)(*@\HLJLni{2}@*)
    (*@\HLJLk{end}@*)
    (*@\HLJLn{ret}@*)
(*@\HLJLk{end}@*)

(*@\HLJLn{n}@*) (*@\HLJLoB{=}@*) (*@\HLJLni{100}@*)
(*@\HLJLn{x}@*) (*@\HLJLoB{=}@*) (*@\HLJLnf{randn}@*)(*@\HLJLp{(}@*)(*@\HLJLn{n}@*)(*@\HLJLp{)}@*)
(*@\HLJLnf{sumsq}@*)(*@\HLJLp{(}@*)(*@\HLJLn{x}@*)(*@\HLJLp{)}@*)
\end{lstlisting}

\begin{lstlisting}
118.10555092168023
\end{lstlisting}


Each step of this algorithm consists of one memory look-up (\texttt{z = x[k]}), one multiplication (\texttt{w = z*z}) and one addition (\texttt{ret = ret + w}). We will ignore the memory look-up in the following discussion. The number of CPU operations per step is therefore 2 (the addition and multiplication). Thus the total number of CPU operations is $2n$. But the constant $2$ here is misleading: we didn't count the memory look-up, thus it is more sensible to just talk about the asymptotic complexity, that is, the \emph{computational cost} is $O(n)$.

Now consider a double sum like:

\[
\sum_{k=1}^n \sum_{j=1}^k x_j^2
\]
which we might implement as:


\begin{lstlisting}
(*@\HLJLk{function}@*) (*@\HLJLnf{sumsq2}@*)(*@\HLJLp{(}@*)(*@\HLJLn{x}@*)(*@\HLJLp{)}@*)
    (*@\HLJLn{n}@*) (*@\HLJLoB{=}@*) (*@\HLJLnf{length}@*)(*@\HLJLp{(}@*)(*@\HLJLn{x}@*)(*@\HLJLp{)}@*)
    (*@\HLJLn{ret}@*) (*@\HLJLoB{=}@*) (*@\HLJLnfB{0.0}@*)
    (*@\HLJLk{for}@*) (*@\HLJLn{k}@*) (*@\HLJLoB{=}@*) (*@\HLJLni{1}@*)(*@\HLJLoB{:}@*)(*@\HLJLn{n}@*)
        (*@\HLJLk{for}@*) (*@\HLJLn{j}@*) (*@\HLJLoB{=}@*) (*@\HLJLni{1}@*)(*@\HLJLoB{:}@*)(*@\HLJLn{k}@*)
            (*@\HLJLn{ret}@*) (*@\HLJLoB{=}@*) (*@\HLJLn{ret}@*) (*@\HLJLoB{+}@*) (*@\HLJLn{x}@*)(*@\HLJLp{[}@*)(*@\HLJLn{j}@*)(*@\HLJLp{]}@*)(*@\HLJLoB{{\textasciicircum}}@*)(*@\HLJLni{2}@*)
        (*@\HLJLk{end}@*)
    (*@\HLJLk{end}@*)
    (*@\HLJLn{ret}@*)
(*@\HLJLk{end}@*)

(*@\HLJLn{n}@*) (*@\HLJLoB{=}@*) (*@\HLJLni{100}@*)
(*@\HLJLn{x}@*) (*@\HLJLoB{=}@*) (*@\HLJLnf{randn}@*)(*@\HLJLp{(}@*)(*@\HLJLn{n}@*)(*@\HLJLp{)}@*)
(*@\HLJLnf{sumsq2}@*)(*@\HLJLp{(}@*)(*@\HLJLn{x}@*)(*@\HLJLp{)}@*)
\end{lstlisting}

\begin{lstlisting}
4730.625569045884
\end{lstlisting}


Now the inner loop is $O(1)$ operations (we don't try to count the precise number), which we do $k$ times for $O(k)$ operations as $k \ensuremath{\rightarrow} \ensuremath{\infty}$. The outer loop therefore takes

\[
\ensuremath{\sum}_{k = 1}^n O(k) = O\left(\ensuremath{\sum}_{k = 1}^n k\right) = O\left( {n (n+1) \over 2} \right) = O(n^2)
\]
operations.



\end{document}
